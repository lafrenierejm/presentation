\section{M1}

\begin{frame}{Specification}
  \justifying
  For each \alert<2>{field} dereference, such as \java{o.f}, add a conditional checker to perform the field dereference only when \java{o != null}.
\end{frame}

\begin{frame}[fragile]{ASMified sample}
  \begin{columns}
    \begin{column}{0.35\textwidth}
      \begin{Java}[%
        % frame=single,%
        % numbers=left,%
        % xleftmargin=1em,%
        ]
        Rct r = new Rct(1, 2);
        int width;

        width = r.width;
      \end{Java}
    \end{column}
    \begin{column}{0.60\textwidth}
      \small
      \begin{lstlisting}[%
          % frame=single,%
          autogobble=true,%
          columns=fullflexible,%
          language={},%
          % numbers=left,%
          % xleftmargin=0.5em,%
        ]
        visitTypeInsn(NEW, "Rct");
        visitInsn(DUP);
        visitInsn(ICONST_2);
        visitInsn(ICONST_3);
        visitMethodInsn(INVOKESPECIAL... "<init>"...);
        visitVarInsn(ASTORE, 1);
        visitVarInsn(ALOAD, 1);
        visitFieldInsn(GETFIELD, "Rct", "width", "I");
        visitVarInsn(ISTORE, 2);
    \end{lstlisting}
    \end{column}
  \end{columns}
\end{frame}

\begin{frame}[fragile]{ASMified sample}
  \begin{columns}
    \begin{column}{0.35\textwidth}
      \begin{Java}[%
          % frame=single,%
          % numbers=left,%
          % linebackgroundsep=1em,%
          linebackgroundcolor={%
            \ifnum 0<\value{lstnumber} \color{white}\fi%
            \ifnum 4=\value{lstnumber} \color{mLightGreen}\fi%
            % \ifnum 3<\value{lstnumber} \color{red}\fi,%
          },%
        ]
        Rct r = new Rct(1, 2);
        int width;

        if (null != r)
            width = r.width;
      \end{Java}
    \end{column}
    \begin{column}{0.60\textwidth}
      \small
      \begin{lstlisting}[%
          % frame=single,%
          autogobble=true,%
          columns=fullflexible,%
          % numbers=left,%
          language={},%
          % breaklines=true,%
          % linebackgroundsep=2.25em,%
          linebackgroundcolor={%
            \ifnum 0<\value{lstnumber} \color{white}\fi%
            \ifnum 6=\value{lstnumber} \color{mLightGreen}\fi%
            \ifnum 7=\value{lstnumber} \color{mLightGreen}\fi%
            \ifnum 8=\value{lstnumber} \color{mLightGreen}\fi%
            \ifnum 9=\value{lstnumber} \color{mLightGreen}\fi%
            \ifnum 10=\value{lstnumber} \color{mLightGreen}\fi%
            \ifnum 14=\value{lstnumber} \color{mLightGreen}\fi%
            \ifnum 15=\value{lstnumber} \color{mLightGreen}\fi%
            % \ifnum 3<\value{lstnumber} \color{red}\fi%
          },%
          % numbers=left,%
          % xleftmargin=0.5em,%
        ]
        visitTypeInsn(NEW, "Rct");
        visitInsn(DUP);
        visitInsn(ICONST_1);
        visitInsn(ICONST_2);
        visitMethodInsn(INVOKESPECIAL... "<init>"...);
        visitVarInsn(ASTORE, 1);
        visitInsn(ACONST_NULL);
        visitVarInsn(ALOAD, 1);
        el l0 = new Label();
        visitJumpInsn(IF_ACMPEQ, l0);
        visitVarInsn(ALOAD, 1);
        visitFieldInsn(GETFIELD, "Rct", "width", "I");
        visitVarInsn(ISTORE, 2);
        visitLabel(l0);
        visitFrame(Opcodes.F_APPEND,1, ...);
      \end{lstlisting}
    \end{column}
  \end{columns}
\end{frame}

\begin{frame}{Walking without jumping}
  \begin{columns}[T]
    \begin{column}{0.35\textwidth}
      \begin{overprint}
        \onslide<1|only@1>\begin{drawstack}
          \cell{objectref \textrightarrow{} r}
        \end{drawstack}
        \onslide<2|only@2>\begin{drawstack}
          \cell{objectref \textrightarrow{} r}
          \cell{objectref \textrightarrow{} r}
        \end{drawstack}
        \onslide<3|only@3>\begin{drawstack}
          \cell{objectref \textrightarrow{} r}
          \cell{objectref \textrightarrow{} r}
          \cell{1}
        \end{drawstack}
        \onslide<4|only@4>\begin{drawstack}
          \cell{objectref \textrightarrow{} r}
          \cell{objectref \textrightarrow{} r}
          \cell{1}
          \cell{2}
        \end{drawstack}
        \onslide<5|only@5>\begin{drawstack}
          \cell{objectref \textrightarrow{} r}
          \cell{objectref \textrightarrow{} r}
        \end{drawstack}
        \onslide<6|only@6>\begin{drawstack}
          \cell{objectref \textrightarrow{} r}
        \end{drawstack}
        \onslide<7|only@7>\begin{drawstack}
          \cell{objectref \textrightarrow{} r}
          \cell{objectref \textrightarrow{} \java{null}}
        \end{drawstack}
        \onslide<8-9|only@8-9>\begin{drawstack}
          \cell{objectref \textrightarrow{} r}
          \cell{objectref \textrightarrow{} \java{null}}
          \cell{objectref \textrightarrow{} r}
        \end{drawstack}
        \onslide<10|only@10>\begin{drawstack}
          \cell{objectref \textrightarrow{} r}
        \end{drawstack}
        \onslide<11|only@11>\begin{drawstack}
          \cell{objectref \textrightarrow{} r}
          \cell{objectref \textrightarrow{} r}
        \end{drawstack}
        \onslide<12|only@12>\begin{drawstack}
          \cell{objectref \textrightarrow{} r}
          \cell{1}
        \end{drawstack}
        \onslide<13-|only@13->\begin{drawstack}
          \cell{objectref \textrightarrow{} r}
        \end{drawstack}
      \end{overprint}
    \end{column}
    \begin{column}{0.60\textwidth}
      \small
      \setlength{\leftmargin}{0pt}
      \setlength{\leftmargini}{0pt}
      \lstset{columns=fullflexible}
      \begin{itemizeStepWithAlert}
        \setlength{\itemsep}{0pt}
        \setlength{\parskip}{0pt}
        \setlength{\parsep}{0pt}
      \item[] % 1
        \lstinline{visitTypeInsn(NEW, "Rct");}
      \item[] % 2
        \lstinline{visitInsn(DUP);}
      \item[] % 3
        \lstinline{visitInsn(ICONST_1);}
      \item[] % 4
        \lstinline{visitInsn(ICONST_2);}
      \item[] % 5
        \lstinline{visitMethodInsn(INVOKESPECIAL... "<init>"...);}
      \item[] % 6
        \lstinline{visitVarInsn(ASTORE, 1);}
      \item[] % 7
        \lstinline{visitInsn(ACONST_NULL);}
      \item[] % 8
        \lstinline{visitVarInsn(ALOAD, 1);}
      \item[] % 9
        \lstinline{el l0 = new Label();}
      \item[] % 10
        \lstinline{visitJumpInsn(IF_ACMPEQ, l0);}
      \item[] % 11
        \lstinline{visitVarInsn(ALOAD, 1);}
      \item[] % 12
        \lstinline{visitFieldInsn(GETFIELD, "Rct", "width", "I");}
      \item[] % 13
        \lstinline{visitVarInsn(ISTORE, 2);}
      \item[] % 14
        \lstinline{visitLabel(l0);}
      \item[] % 15
        \lstinline{visitFrame(Opcodes.F_APPEND,1, ...);}
      \end{itemizeStepWithAlert}
    \end{column}
  \end{columns}
\end{frame}

\begin{frame}{PIT implementation}
  Override \java{visitFieldInsn()} for \lstinline{PUTFIELD} and \lstinline{GETFIELD}:
  \begin{enumerate}
  \item<alert@2>
    Duplicate the objectref on top of the stack
  \item<alert@3>
    Push a \java{null} reference onto the stack
  \item<alert@4>
    Compare the references' equality
  \end{enumerate}
  \begin{columns}[t,totalwidth=0.6\textwidth]
    \begin{column}{0.45\textwidth}
      \alert<5-7>{Equal (\java{null}):}
      \begin{enumerate}
        \setcounter{enumi}{3}
      \item<alert@5>
        Skip the original opcode
      \item<alert@6>
        Pop the unneeded object reference from the stack
      \item<alert@7>
        Return to normal execution
      \end{enumerate}
    \end{column}
    \begin{column}[]{0.45\textwidth}
      \alert<8-9>{Not equal:}
      \begin{enumerate}
        \setcounter{enumi}{3}
      \item<alert@8>
        Execute the original field access opcode
      \item<alert@9>
        Return to normal execution
      \end{enumerate}
    \end{column}
  \end{columns}
\end{frame}

%%% Local Variables:
%%% mode: latex
%%% TeX-master: "../presentation"
%%% End:
