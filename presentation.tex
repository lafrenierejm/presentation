\documentclass[10pt]{beamer}

\usetheme[progressbar=frametitle]{metropolis}
\usepackage{appendixnumberbeamer}

\usepackage{xspace}
\newcommand{\themename}{\textbf{\textsc{metropolis}}\xspace}

% encoding and locale
\usepackage[utf8]{inputenc}
\usepackage[american]{babel}
\usepackage{csquotes}
%% English oddities
\makeatletter % changes the catcode of '@' to 11
\newcommand\etal{et al\@ifnextchar.{}{.\@}}
\newcommand\etc{etc\@ifnextchar.{}{.\@}}
\newcommand\ie{i.e.\@}
\newcommand\eg{e.g.\@}
\makeatother % restores the catcode of '@' to 12

% source code
\usepackage{listings}
\renewcommand{\lstlistingname}{}
\renewcommand{\lstlistingname}{}
\usepackage{lstautogobble}
\usepackage{lstlinebgrd}
%% Java environment
\lstnewenvironment{Java}[1][]{%
  \lstset{autogobble=true,columns=fullflexible,language=java,#1}%
}{}
%% Java inline
\newcommand{\java}[1]{%
  \lstinline[autogobble=true,columns=fullflexible,language=java]{#1}
}

% lists
\newenvironment{itemizeStepWithAlert}{\begin{itemize}[<+-| alert@+>]}{\end{itemize}}

% graphics
%% drawing stacks
\usepackage[nocolor]{drawstack}
%% plotting
\usepackage{pgfplots}
\usepgfplotslibrary{dateplot}
%% Creative Commons icons
\usepackage[scale=2]{ccicons}

% tables
\usepackage{booktabs}
\usepackage{tabu}

% hyperlinks
\usepackage{hyperref}
%% \urlanglebracket encloses a URL in angle brackets
\DeclareUrlCommand\urlanglebracket{%
  \def\UrlLeft{<}%
  \def\UrlRight{>}%
}
%% macros
\newcommand{\email}[1]{\href{mailto:#1}{\url{#1}}}
\newcommand{\emailcontact}[1]{\href{mailto:#1}{\urlanglebracket{#1}}}

% document metadata
\newcommand{\titletext}{Augmenting the PIT Mutation Testing Tool}
%% \maketitle information
\title{\titletext}
\subtitle{Implementing Mutation Operators and Code-Fixing Rules}
% \date{\today}
\date{}
\author{%
  \urlstyle{same}
  Khoa Nguyen \emailcontact{kxn161730@utdallas.edu}\\
  Leejia James \emailcontact{lxj171130@utdallas.edu}\\
  Joseph LaFreniere \emailcontact{lafrenierejm@utdallas.edu}
}
\hypersetup{%
  pdfauthor={Nguyen, Khoa and James, Leejia and LaFreniere, Joseph},
  pdftitle={\titletext},
  pdfkeywords={PIT,mutator,mutation,software,testing,JVM,bytecode}
}
\institute{The University of Texas at Dallas}
% \titlegraphic{\hfill\includegraphics[height=1.5cm]{logo.pdf}}

\begin{document}

\maketitle

\begin{frame}{Table of contents}
  \setbeamertemplate{section in toc}[sections numbered]
  \tableofcontents[hideallsubsections]
\end{frame}

\section{M1}

\begin{frame}{Specification}
  \justifying
  For each \alert<2>{field} dereference, such as \java{o.f}, add a conditional checker to perform the field dereference only when \java{o != null}.
\end{frame}

\begin{frame}[fragile]{ASMified sample}
  \begin{columns}
    \begin{column}{0.35\textwidth}
      \begin{Java}[%
        % frame=single,%
        % numbers=left,%
        % xleftmargin=1em,%
        ]
        Rct r = new Rct(1, 2);
        int width;

        width = r.width;
      \end{Java}
    \end{column}
    \begin{column}{0.60\textwidth}
      \small
      \begin{lstlisting}[%
          % frame=single,%
          autogobble=true,%
          columns=fullflexible,%
          language={},%
          % numbers=left,%
          % xleftmargin=0.5em,%
        ]
        visitTypeInsn(NEW, "Rct");
        visitInsn(DUP);
        visitInsn(ICONST_2);
        visitInsn(ICONST_3);
        visitMethodInsn(INVOKESPECIAL... "<init>"...);
        visitVarInsn(ASTORE, 1);
        visitVarInsn(ALOAD, 1);
        visitFieldInsn(GETFIELD, "Rct", "width", "I");
        visitVarInsn(ISTORE, 2);
    \end{lstlisting}
    \end{column}
  \end{columns}
\end{frame}

\begin{frame}[fragile]{ASMified sample}
  \begin{columns}
    \begin{column}{0.35\textwidth}
      \begin{Java}[%
          % frame=single,%
          % numbers=left,%
          % linebackgroundsep=1em,%
          linebackgroundcolor={%
            \ifnum 0<\value{lstnumber} \color{white}\fi%
            \ifnum 4=\value{lstnumber} \color{mLightGreen}\fi%
            % \ifnum 3<\value{lstnumber} \color{red}\fi,%
          },%
        ]
        Rct r = new Rct(1, 2);
        int width;

        if (null != r)
            width = r.width;
      \end{Java}
    \end{column}
    \begin{column}{0.60\textwidth}
      \small
      \begin{lstlisting}[%
          % frame=single,%
          autogobble=true,%
          columns=fullflexible,%
          % numbers=left,%
          language={},%
          % breaklines=true,%
          % linebackgroundsep=2.25em,%
          linebackgroundcolor={%
            \ifnum 0<\value{lstnumber} \color{white}\fi%
            \ifnum 6=\value{lstnumber} \color{mLightGreen}\fi%
            \ifnum 7=\value{lstnumber} \color{mLightGreen}\fi%
            \ifnum 8=\value{lstnumber} \color{mLightGreen}\fi%
            \ifnum 9=\value{lstnumber} \color{mLightGreen}\fi%
            \ifnum 10=\value{lstnumber} \color{mLightGreen}\fi%
            \ifnum 14=\value{lstnumber} \color{mLightGreen}\fi%
            \ifnum 15=\value{lstnumber} \color{mLightGreen}\fi%
            % \ifnum 3<\value{lstnumber} \color{red}\fi%
          },%
          % numbers=left,%
          % xleftmargin=0.5em,%
        ]
        visitTypeInsn(NEW, "Rct");
        visitInsn(DUP);
        visitInsn(ICONST_1);
        visitInsn(ICONST_2);
        visitMethodInsn(INVOKESPECIAL... "<init>"...);
        visitVarInsn(ASTORE, 1);
        visitInsn(ACONST_NULL);
        visitVarInsn(ALOAD, 1);
        el l0 = new Label();
        visitJumpInsn(IF_ACMPEQ, l0);
        visitVarInsn(ALOAD, 1);
        visitFieldInsn(GETFIELD, "Rct", "width", "I");
        visitVarInsn(ISTORE, 2);
        visitLabel(l0);
        visitFrame(Opcodes.F_APPEND,1, ...);
      \end{lstlisting}
    \end{column}
  \end{columns}
\end{frame}

\begin{frame}{Walking without jumping}
  \begin{columns}[T]
    \begin{column}{0.35\textwidth}
      \begin{overprint}
        \onslide<1|only@1>\begin{drawstack}
          \cell{objectref \textrightarrow{} r}
        \end{drawstack}
        \onslide<2|only@2>\begin{drawstack}
          \cell{objectref \textrightarrow{} r}
          \cell{objectref \textrightarrow{} r}
        \end{drawstack}
        \onslide<3|only@3>\begin{drawstack}
          \cell{objectref \textrightarrow{} r}
          \cell{objectref \textrightarrow{} r}
          \cell{1}
        \end{drawstack}
        \onslide<4|only@4>\begin{drawstack}
          \cell{objectref \textrightarrow{} r}
          \cell{objectref \textrightarrow{} r}
          \cell{1}
          \cell{2}
        \end{drawstack}
        \onslide<5|only@5>\begin{drawstack}
          \cell{objectref \textrightarrow{} r}
          \cell{objectref \textrightarrow{} r}
        \end{drawstack}
        \onslide<6|only@6>\begin{drawstack}
          \cell{objectref \textrightarrow{} r}
        \end{drawstack}
        \onslide<7|only@7>\begin{drawstack}
          \cell{objectref \textrightarrow{} r}
          \cell{objectref \textrightarrow{} \java{null}}
        \end{drawstack}
        \onslide<8-9|only@8-9>\begin{drawstack}
          \cell{objectref \textrightarrow{} r}
          \cell{objectref \textrightarrow{} \java{null}}
          \cell{objectref \textrightarrow{} r}
        \end{drawstack}
        \onslide<10|only@10>\begin{drawstack}
          \cell{objectref \textrightarrow{} r}
        \end{drawstack}
        \onslide<11|only@11>\begin{drawstack}
          \cell{objectref \textrightarrow{} r}
          \cell{objectref \textrightarrow{} r}
        \end{drawstack}
        \onslide<12|only@12>\begin{drawstack}
          \cell{objectref \textrightarrow{} r}
          \cell{1}
        \end{drawstack}
        \onslide<13-|only@13->\begin{drawstack}
          \cell{objectref \textrightarrow{} r}
        \end{drawstack}
      \end{overprint}
    \end{column}
    \begin{column}{0.60\textwidth}
      \small
      \setlength{\leftmargin}{0pt}
      \setlength{\leftmargini}{0pt}
      \lstset{columns=fullflexible}
      \begin{itemizeStepWithAlert}
        \setlength{\itemsep}{0pt}
        \setlength{\parskip}{0pt}
        \setlength{\parsep}{0pt}
      \item[] % 1
        \lstinline{visitTypeInsn(NEW, "Rct");}
      \item[] % 2
        \lstinline{visitInsn(DUP);}
      \item[] % 3
        \lstinline{visitInsn(ICONST_1);}
      \item[] % 4
        \lstinline{visitInsn(ICONST_2);}
      \item[] % 5
        \lstinline{visitMethodInsn(INVOKESPECIAL... "<init>"...);}
      \item[] % 6
        \lstinline{visitVarInsn(ASTORE, 1);}
      \item[] % 7
        \lstinline{visitInsn(ACONST_NULL);}
      \item[] % 8
        \lstinline{visitVarInsn(ALOAD, 1);}
      \item[] % 9
        \lstinline{el l0 = new Label();}
      \item[] % 10
        \lstinline{visitJumpInsn(IF_ACMPEQ, l0);}
      \item[] % 11
        \lstinline{visitVarInsn(ALOAD, 1);}
      \item[] % 12
        \lstinline{visitFieldInsn(GETFIELD, "Rct", "width", "I");}
      \item[] % 13
        \lstinline{visitVarInsn(ISTORE, 2);}
      \item[] % 14
        \lstinline{visitLabel(l0);}
      \item[] % 15
        \lstinline{visitFrame(Opcodes.F_APPEND,1, ...);}
      \end{itemizeStepWithAlert}
    \end{column}
  \end{columns}
\end{frame}

%%% Local Variables:
%%% mode: latex
%%% TeX-master: "../presentation"
%%% End:


\section{Introduction}

\begin{frame}[fragile]{Metropolis}

  The \themename theme is a Beamer theme with minimal visual noise
  inspired by the \href{https://github.com/hsrmbeamertheme/hsrmbeamertheme}{\textsc{hsrm} Beamer
  Theme} by Benjamin Weiss.

  Enable the theme by loading

  \begin{verbatim}    \documentclass{beamer}
    \usetheme{metropolis}\end{verbatim}

  Note, that you have to have Mozilla's \emph{Fira Sans} font and XeTeX
  installed to enjoy this wonderful typography.
\end{frame}
\begin{frame}[fragile]{Sections}
  Sections group slides of the same topic

  \begin{verbatim}    \section{Elements}\end{verbatim}

  for which \themename provides a nice progress indicator \ldots
\end{frame}

\section{Titleformats}

\begin{frame}{Metropolis titleformats}
	\themename supports 4 different titleformats:
	\begin{itemize}
		\item Regular
		\item \textsc{Smallcaps}
		\item \textsc{allsmallcaps}
		\item ALLCAPS
	\end{itemize}
	They can either be set at once for every title type or individually.
\end{frame}

{
    \metroset{titleformat frame=smallcaps}
\begin{frame}{Small caps}
	This frame uses the \texttt{smallcaps} titleformat.

	\begin{alertblock}{Potential Problems}
		Be aware, that not every font supports small caps. If for example you typeset your presentation with pdfTeX and the Computer Modern Sans Serif font, every text in smallcaps will be typeset with the Computer Modern Serif font instead.
	\end{alertblock}
\end{frame}
}

{
\metroset{titleformat frame=allsmallcaps}
\begin{frame}{All small caps}
	This frame uses the \texttt{allsmallcaps} titleformat.

	\begin{alertblock}{Potential problems}
		As this titleformat also uses smallcaps you face the same problems as with the \texttt{smallcaps} titleformat. Additionally this format can cause some other problems. Please refer to the documentation if you consider using it.

		As a rule of thumb: Just use it for plaintext-only titles.
	\end{alertblock}
\end{frame}
}

{
\metroset{titleformat frame=allcaps}
\begin{frame}{All caps}
	This frame uses the \texttt{allcaps} titleformat.

	\begin{alertblock}{Potential Problems}
		This titleformat is not as problematic as the \texttt{allsmallcaps} format, but basically suffers from the same deficiencies. So please have a look at the documentation if you want to use it.
	\end{alertblock}
\end{frame}
}

\section{Elements}

\begin{frame}[fragile]{Typography}
      \begin{verbatim}The theme provides sensible defaults to
\emph{emphasize} text, \alert{accent} parts
or show \textbf{bold} results.\end{verbatim}

  \begin{center}becomes\end{center}

  The theme provides sensible defaults to \emph{emphasize} text,
  \alert{accent} parts or show \textbf{bold} results.
\end{frame}

\begin{frame}{Font feature test}
  \begin{itemize}
    \item Regular
    \item \textit{Italic}
    \item \textsc{SmallCaps}
    \item \textbf{Bold}
    \item \textbf{\textit{Bold Italic}}
    \item \textbf{\textsc{Bold SmallCaps}}
    \item \texttt{Monospace}
    \item \texttt{\textit{Monospace Italic}}
    \item \texttt{\textbf{Monospace Bold}}
    \item \texttt{\textbf{\textit{Monospace Bold Italic}}}
  \end{itemize}
\end{frame}

\begin{frame}{Lists}
  \begin{columns}[T,onlytextwidth]
    \column{0.33\textwidth}
      Items
      \begin{itemize}
        \item Milk \item Eggs \item Potatos
      \end{itemize}

    \column{0.33\textwidth}
      Enumerations
      \begin{enumerate}
        \item First, \item Second and \item Last.
      \end{enumerate}

    \column{0.33\textwidth}
      Descriptions
      \begin{description}
        \item[PowerPoint] Meeh. \item[Beamer] Yeeeha.
      \end{description}
  \end{columns}
\end{frame}
\begin{frame}{Animation}
  \begin{itemize}[<+- | alert@+>]
    \item \alert<4>{This is\only<4>{ really} important}
    \item Now this
    \item And now this
  \end{itemize}
\end{frame}
\begin{frame}{Figures}
  \begin{figure}
    \newcounter{density}
    \setcounter{density}{20}
    \begin{tikzpicture}
      \def\couleur{alerted text.fg}
      \path[coordinate] (0,0)  coordinate(A)
                  ++( 90:5cm) coordinate(B)
                  ++(0:5cm) coordinate(C)
                  ++(-90:5cm) coordinate(D);
      \draw[fill=\couleur!\thedensity] (A) -- (B) -- (C) --(D) -- cycle;
      \foreach \x in {1,...,40}{%
          \pgfmathsetcounter{density}{\thedensity+20}
          \setcounter{density}{\thedensity}
          \path[coordinate] coordinate(X) at (A){};
          \path[coordinate] (A) -- (B) coordinate[pos=.10](A)
                              -- (C) coordinate[pos=.10](B)
                              -- (D) coordinate[pos=.10](C)
                              -- (X) coordinate[pos=.10](D);
          \draw[fill=\couleur!\thedensity] (A)--(B)--(C)-- (D) -- cycle;
      }
    \end{tikzpicture}
    \caption{Rotated square from
    \href{http://www.texample.net/tikz/examples/rotated-polygons/}{texample.net}.}
  \end{figure}
\end{frame}
\begin{frame}{Tables}
  \begin{table}
    \caption{Largest cities in the world (source: Wikipedia)}
    \begin{tabular}{lr}
      \toprule
      City & Population\\
      \midrule
      Mexico City & 20,116,842\\
      Shanghai & 19,210,000\\
      Peking & 15,796,450\\
      Istanbul & 14,160,467\\
      \bottomrule
    \end{tabular}
  \end{table}
\end{frame}
\begin{frame}{Blocks}
  Three different block environments are pre-defined and may be styled with an
  optional background color.

  \begin{columns}[T,onlytextwidth]
    \column{0.5\textwidth}
      \begin{block}{Default}
        Block content.
      \end{block}

      \begin{alertblock}{Alert}
        Block content.
      \end{alertblock}

      \begin{exampleblock}{Example}
        Block content.
      \end{exampleblock}

    \column{0.5\textwidth}

      \metroset{block=fill}

      \begin{block}{Default}
        Block content.
      \end{block}

      \begin{alertblock}{Alert}
        Block content.
      \end{alertblock}

      \begin{exampleblock}{Example}
        Block content.
      \end{exampleblock}

  \end{columns}
\end{frame}
\begin{frame}{Math}
  \begin{equation*}
    e = \lim_{n\to \infty} \left(1 + \frac{1}{n}\right)^n
  \end{equation*}
\end{frame}
\begin{frame}{Line plots}
  \begin{figure}
    \begin{tikzpicture}
      \begin{axis}[
        mlineplot,
        width=0.9\textwidth,
        height=6cm,
      ]

        \addplot {sin(deg(x))};
        \addplot+[samples=100] {sin(deg(2*x))};

      \end{axis}
    \end{tikzpicture}
  \end{figure}
\end{frame}
\begin{frame}{Bar charts}
  \begin{figure}
    \begin{tikzpicture}
      \begin{axis}[
        mbarplot,
        xlabel={Foo},
        ylabel={Bar},
        width=0.9\textwidth,
        height=6cm,
      ]

      \addplot plot coordinates {(1, 20) (2, 25) (3, 22.4) (4, 12.4)};
      \addplot plot coordinates {(1, 18) (2, 24) (3, 23.5) (4, 13.2)};
      \addplot plot coordinates {(1, 10) (2, 19) (3, 25) (4, 15.2)};

      \legend{lorem, ipsum, dolor}

      \end{axis}
    \end{tikzpicture}
  \end{figure}
\end{frame}
\begin{frame}{Quotes}
  \begin{quote}
    Veni, Vidi, Vici
  \end{quote}
\end{frame}

{%
\setbeamertemplate{frame footer}{My custom footer}
\begin{frame}[fragile]{Frame footer}
    \themename defines a custom beamer template to add a text to the footer. It can be set via
    \begin{verbatim}\setbeamertemplate{frame footer}{My custom footer}\end{verbatim}
\end{frame}
}

\begin{frame}{References}
  Some references to showcase [allowframebreaks] \cite{knuth92,ConcreteMath,Simpson,Er01,greenwade93}
\end{frame}

\section{Conclusion}

\begin{frame}{Summary}

  Get the source of this theme and the demo presentation from

  \begin{center}\url{github.com/matze/mtheme}\end{center}

  The theme \emph{itself} is licensed under a
  \href{http://creativecommons.org/licenses/by-sa/4.0/}{Creative Commons
  Attribution-ShareAlike 4.0 International License}.

  \begin{center}\ccbysa\end{center}

\end{frame}

{\setbeamercolor{palette primary}{fg=black, bg=yellow}
\begin{frame}[standout]
  Questions?
\end{frame}
}

\appendix

\begin{frame}[fragile]{Backup slides}
  Sometimes, it is useful to add slides at the end of your presentation to
  refer to during audience questions.

  The best way to do this is to include the \verb|appendixnumberbeamer|
  package in your preamble and call \verb|\appendix| before your backup slides.

  \themename will automatically turn off slide numbering and progress bars for
  slides in the appendix.
\end{frame}

\begin{frame}[allowframebreaks]{References}

  \bibliography{demo}
  \bibliographystyle{abbrv}

\end{frame}

\end{document}
